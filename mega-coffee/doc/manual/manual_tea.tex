\documentclass[a4paper,10pt]{scrartcl}
\usepackage[utf8x]{inputenc}
\usepackage[pdfauthor={Carsten Kemena},colorlinks=true,linkcolor=blue, pdftitle={MSAA - Manual}]{hyperref}


% Title Page
\title{tea v1.0 - Manual}
\author{Carsten Kemena}
\date{\today}

\begin{document}
\maketitle

% \begin{abstract}
% \end{abstract}

\section{Introduction}
Tea is a simple multiple sequence aligner. It is a re-implementation of the T-Coffee program. 

\section{Options}
The MSAA program distinguished between three kinds of options:
\begin{itemize}
	\item \textbf{General options:} These options regulate the input and output behavior of MSAA - the input/output files and formats.
	\item \textbf{Pairwise alignment options:} These options regulate how the pairwise alignment library is generated.
	\item \textbf{Guide tree options:} These options change the computation of the guide tree.
\end{itemize}



\subsection{General options}
There are several general options available:

\textbf{i,in}: The input sequence file in FASTA format.

\textbf{o,out}: The output file for the computed alignment.

\textbf{f,format}: The format which should be used to write the alignment. Currently supported: fasta, clustalw, msf, phylip\_i (Phylip interleaved), phylip\_s (Phylip sequential).

\textbf{n,n\_threads}: The number of threads to use to compute the alignment. Should be a number smaller or equal to the number of cores to use.

\subsection{Pairwise alignment options}

\subsection{Guide tree options}

\end{document}          
